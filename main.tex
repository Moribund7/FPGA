\documentclass{article}
\usepackage[utf8]{inputenc}

\title{Przetwarzanie obrazu w czasie rzeczywistym}
\author{Filip Kowalski}
\date{29 Maja 2018}

\usepackage{natbib}
\usepackage{graphicx}
\usepackage{polski}
\usepackage{url}

\newcommand{\tylda }{{\raise.17ex\hbox{$\scriptstyle\sim$}}}

\begin{document}

\maketitle

\section*{Wprowadzenie}
Celem projektu było przetestowanie możliwości układu Cyclone V do przetwarzania obrazu w czasie rzeczywistym. W tym celu wykorzystano w/w układ oraz kamerę ,,KELIMA SQ10 Mini 1080P HD DVR'' (dzieło chińskich rączek -- zakupiona na portalu gearbest.com za \tylda 5 \$). W projekcie wykorzystano przykład pochodzący z dokumentacji układu -- link do pakietu \cite{przyklad}. Kamera była podpinana do wejścia VIDEO\_IN a następnie wyświetlana na monitorze przy pomocy wyjścia VGA. 

\section*{Realizacja}
Kod pochodzący z przykładów z dokumentacji został tak zmodyfikowany, aby sprawdzić różne sposoby obróbki obrazu. W tym celu stworzono kilka filtrów, wybieranych  przy pomocy przełączników SW[9:6], i których parametry można ustawić przy pomocy pozostałych przełączników. Poniżej zamieszczono instrukcję opisującą sposób uruchamiania danego filtra (układ przełączników jakie należy ustawić), sposób jego działania oraz parametry jakie przyjmuje.

\section*{Instrukcja obsługi}
\begin{itemize}
    

\item SW[9] and SW[8]
 wykrywanie kolorów określonych przy pomocy trzech parametrów: dzielnik - int ustalany przy kompilacji pliku( ze względu
na zbyt małą liczbę przełączników do ustalania parametrów) oraz dwa inty przy pomocy SW[5-3] drugi SW[2-0] - kontrolują kolory,
Przykładowe ustawienia: 
--dzielnik=3,SW[3],SW[2] ustawienie wykrywania koloru człowieka


\item SW[9] and $\neg$SW[8] zamienia kolory parami przy pomocy przełączników 2-0


\item SW[8] and $\neg$W[9] 
ustawia kolejne kombinacje kolorów (wszystkich jest 27): na każdy z trzech kolorów na wyjściu ustawiamy jeden z trzech kolorów na wejściu. Kolejne kombinacje wywołuje się przy pomocy ustawienia odpowiedniej wartości za pomocą przełączników 4-0


\item SW[7] and $\neg$SW[6]
wyświetla cztery obrazy każdy z inną jasnością


\item $\neg$SW[7] and SW[6]
wyświetla cztery obrazy każdy bez jednego podstawowego koloru


\item SW[7] and SW[6]
wyświetla cztery obrazy -- w każdym wybiera filtrowany  kolor, ustawiony przy pomocy przełączników SW[5-0], a następnie wycina wszystkie inne kolory poza tym wybranym. Wybrany w ten sposób kolor jest wyświetlany nie za pomocą swojego rzeczywistego koloru tylko jako jeden z trzech podstawowych kolorów RGB

\end{itemize}



\begin{thebibliography}{1}

  \bibitem{przyklad} 
  \url{http://www.fuw.edu.pl/~pablo/fpga/get.php?file=laboratorium/de1soc/docs/de1-soc/Demonstrations/FPGA/DE1_SoC_TV.zip}

 

  \end{thebibliography}

 
\end{document}
